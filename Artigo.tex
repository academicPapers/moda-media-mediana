\documentclass[a4paper]{article}
\usepackage{lmodern}
\usepackage{amssymb,amsmath}
\usepackage{ifxetex,ifluatex}
\usepackage{fixltx2e} % provides \textsubscript
\ifnum 0\ifxetex 1\fi\ifluatex 1\fi=0 % if pdftex
  \usepackage[T1]{fontenc}
  \usepackage[utf8]{inputenc}
\else % if luatex or xelatex
  \ifxetex
    \usepackage{mathspec}
  \else
    \usepackage{fontspec}
  \fi
  \defaultfontfeatures{Ligatures=TeX,Scale=MatchLowercase}
\fi
% use upquote if available, for straight quotes in verbatim environments
\IfFileExists{upquote.sty}{\usepackage{upquote}}{}
% use microtype if available
\IfFileExists{microtype.sty}{%
\usepackage{microtype}
\UseMicrotypeSet[protrusion]{basicmath} % disable protrusion for tt fonts
}{}
\usepackage[margin=1in]{geometry}
\usepackage{hyperref}
\hypersetup{unicode=true,
            pdftitle={Avaliação pela Moda, Média ou Mediana?},
            pdfauthor={Luiz Fernando Palin Droubi; Norberto Hochheim; Willian Zonato},
            pdfborder={0 0 0},
            breaklinks=true}
\urlstyle{same}  % don't use monospace font for urls
\usepackage{longtable,booktabs}
\usepackage{graphicx,grffile}
\makeatletter
\def\maxwidth{\ifdim\Gin@nat@width>\linewidth\linewidth\else\Gin@nat@width\fi}
\def\maxheight{\ifdim\Gin@nat@height>\textheight\textheight\else\Gin@nat@height\fi}
\makeatother
% Scale images if necessary, so that they will not overflow the page
% margins by default, and it is still possible to overwrite the defaults
% using explicit options in \includegraphics[width, height, ...]{}
\setkeys{Gin}{width=\maxwidth,height=\maxheight,keepaspectratio}
\IfFileExists{parskip.sty}{%
\usepackage{parskip}
}{% else
\setlength{\parindent}{0pt}
\setlength{\parskip}{6pt plus 2pt minus 1pt}
}
\setlength{\emergencystretch}{3em}  % prevent overfull lines
\providecommand{\tightlist}{%
  \setlength{\itemsep}{0pt}\setlength{\parskip}{0pt}}
\setcounter{secnumdepth}{5}
% Redefines (sub)paragraphs to behave more like sections
\ifx\paragraph\undefined\else
\let\oldparagraph\paragraph
\renewcommand{\paragraph}[1]{\oldparagraph{#1}\mbox{}}
\fi
\ifx\subparagraph\undefined\else
\let\oldsubparagraph\subparagraph
\renewcommand{\subparagraph}[1]{\oldsubparagraph{#1}\mbox{}}
\fi

%%% Use protect on footnotes to avoid problems with footnotes in titles
\let\rmarkdownfootnote\footnote%
\def\footnote{\protect\rmarkdownfootnote}

%%% Change title format to be more compact
\usepackage{titling}

% Create subtitle command for use in maketitle
\newcommand{\subtitle}[1]{
  \posttitle{
    \begin{center}\large#1\end{center}
    }
}

\setlength{\droptitle}{-2em}
  \title{Avaliação pela Moda, Média ou Mediana?}
  \pretitle{\vspace{\droptitle}\centering\huge}
  \posttitle{\par}
\subtitle{Teoria e simulações}
  \author{Luiz Fernando Palin Droubi\footnote{SPU/SC,
  \href{mailto:luiz.droubi@planejamento.gov.br}{\nolinkurl{luiz.droubi@planejamento.gov.br}}} \\ Norberto Hochheim\footnote{UFSC,
  \href{mailto:hochheim@gmail.com}{\nolinkurl{hochheim@gmail.com}}} \\ Willian Zonato\footnote{SPU/SC,
  \href{mailto:willian.zonato@planejamento.gov.br}{\nolinkurl{willian.zonato@planejamento.gov.br}}}}
  \preauthor{\centering\large\emph}
  \postauthor{\par}
  \predate{\centering\large\emph}
  \postdate{\par}
  \date{27/04/2018}

\usepackage{booktabs}
\usepackage{longtable}
\usepackage{array}
\usepackage{multirow}
\usepackage[table]{xcolor}
\usepackage{wrapfig}
\usepackage{float}
\usepackage{colortbl}
\usepackage{pdflscape}
\usepackage{tabu}
\usepackage{threeparttable}
\usepackage[normalem]{ulem}

\usepackage[brazil]{babel}
\usepackage{graphicx}
\usepackage{float}
\usepackage{subfig}
\usepackage{caption}

\begin{document}
\maketitle

\section{INTRODUÇÃO}\label{introducao}

Existe na área da avaliação de imóveis uma discussão frequente e
indesejável a respeito da adoção da estimativa de tendência central
adotada para a predição de valores quando da utilização de modelos
lineares log-normais, isto é, modelos em que a variável resposta aparece
transformada pela função logaritmo natural.\footnote{Neste artigo esta
  função é representada por \(log\).}

Pretende-se com este artigo dar a este problema de uma abordagem formal.
Entendemos que a norma brasileira (ABNT,
\protect\hyperlink{ref-NBR1465302}{2011}) deveria tratar este assunto de
maneira clara, especificando qual estimador deveria ser utilizado para a
formação de valores.

\begin{quote}
Major Point 1: When we talk about the relationship of one variable to
one or more others, we are referring to the regression function, which
expresses the mean of the first variable as a function of the others.
The key word here is \emph{mean}! {[}matloff2009, 386, grifo do autor{]}
\end{quote}

\section{REVISÃO BIBLIOGRÁFICA}\label{revisao-bibliografica}

\begin{quote}
Earlier, we often referred to certain estimators as being ``natural.''
For example, if we are estimating a population mean, an obvious choice
of estimator would be the sample mean. But in many applications, it is
less clear what a ``natural'' estimate for a population quantity of
interest would be. We will present general methods for estimation in
this section. We will also discuss advanced methods of inference
(MATLOFF, \protect\hyperlink{ref-matloff2009}{2009}, p. 303).
\end{quote}

\subsection{Estimadores}\label{estimadores}

A definição de um \emph{estimador} para um parâmetro ou uma variável
\(\theta\) é uma função \(\hat{\theta}(X)\), que mapeia o espaço
amostral para um conjunto de estimativas amostrais, em que \(X\) é uma
variável aleatória dos dados observados. É usual denotar uma estimativa
em para um determinado ponto \(x \in X\) por \(\hat{\theta}(X = x)\) ou,
mais simplesmente, \(\hat{\theta}(x)\).

\subsubsection{Propriedades de um
estimador}\label{propriedades-de-um-estimador}

\paragraph{Erro}\label{erro}

\[e(x) = \hat{\theta}(x) - \theta\]

\paragraph{Erro médio quadrático}\label{erro-medio-quadratico}

\[MSE = E[\hat{\theta}(X) - \theta]\]

\paragraph{Desvio}\label{desvio}

\[d(x) = \hat{\theta}(x) - E(\hat{\theta}(X))\] onde
\(E(\hat{\theta}(X))\) é o Valor Esperado do estimador.

\paragraph{Variância}\label{variancia}

\[var(\hat{\theta}) = E[(\hat{\theta} - E(\hat{\theta})^2]\]

\paragraph{Viés}\label{vies}

\[B(\hat{\theta}) = E(\hat{\theta}) - \theta\]

O viés coincide com o valor esperado do erro, pois
\(E(\hat{\theta}) - \theta = E(\hat{\theta}-\theta)\).

\paragraph{Consistência}\label{consistencia}

\[\lim_{n \rightarrow \infty}\hat{\theta} = \theta\]

\subsection{Melhor estimador linear não-inviesado ou BLUE (Best Linear
Unbiased
Estimator)}\label{melhor-estimador-linear-nao-inviesado-ou-blue-best-linear-unbiased-estimator}

Em estatística, é comum o uso da sigla BLUE para indicar o melhor
estimador linear não-enviesado.

\subsection{Regressão linear}\label{regressao-linear}

\subsubsection{Definição precisa}\label{definicao-precisa}

Sejam Y e X duas variáveis e \(m_{Y;X}(t)\) uma função tal que:

\[m_{Y;X}(t) = E(Y|X = t)\]

Chamamos \(m_{Y;X}\) de \textbf{função de regressão de \(Y\) dado \(X\)}
(MATLOFF, \protect\hyperlink{ref-matloff2009}{2009}, p. 386, grifo do
autor). Em geral, \(m_{Y;X}(t)\) é a \textbf{média} da de \(Y\) para
todas as unidades da população para as quais \(X = t\) (MATLOFF,
\protect\hyperlink{ref-matloff2009}{2009}, p. 386, grifo nosso).

Segundo Matloff (\protect\hyperlink{ref-matloff2009}{2009}, p. 386,
grifo do autor), ainda, a função \(m_{Y;X}(t)\) é uma função da
\textbf{população}, ou seja, apenas \textbf{estimamos} uma equação de
regressão (\(\hat{m}_{Y;X}(t)\)) à partir de uma amostra da população.

\begin{quote}
The function \(m_{Y;X}(t)\) is a population entity, so we must estimate
it from our sample data. To do this, we have a choice of either assuming
that \(m_{Y;X}(t)\) takes on some parametric form, or making no such
assumption. If we opt for a parametric approach, the most common model
is linear {[}\ldots{}{]} (MATLOFF,
\protect\hyperlink{ref-matloff2009}{2009}, p. 389).
\end{quote}

Segundo Matloff (\protect\hyperlink{ref-matloff2009}{2009}, pp.
394--397), as proposições acima sobre a função \(m_{Y;X}\) pode ser
generalizada para outras quantidades de regressores em \(X\) e seus
termos de interação, tal que:

\[m_{Y;X}(t) = \beta_0 + \beta_1t_1 + \beta_2t_2 + \beta_3t_1t_2 + \beta_4t_1^2\]

Notando que o termo \textbf{regressão linear} não necessariamente
significa que o gráfico da função de regressão seja uma linha reta ou um
plano, mas que se refere a função de regressão ser linear em relação aos
seus parâmetros (\(\beta_i\)).

\subsection{Estimação em modelos de regressão
paramétricos}\label{estimacao-em-modelos-de-regressao-parametricos}

Segundo Matloff (\protect\hyperlink{ref-matloff2009}{2009}, p. 389), é
possível demonstrar que o mínimo valor da quantidade
\(E[(Y - g(X))^2]\)\footnote{Erro médio quadrático de predição} é
obtido, entre todas as outras funções, para \(g(X) = m_{Y;X}(X)\).
Porém, ``se pretendemos minimizar o erro médio absoluto de predição,
\(E(|Y - g(X)|)\) , a melhor função seria a mediana
\(g(Y) = mediana(Y|X)\).'' (MATLOFF,
\protect\hyperlink{ref-matloff2009}{2009}, p. 389).

\subsection{Esperança matemática ou Valor
Esperado}\label{esperanca-matematica-ou-valor-esperado}

Segundo WIKIPEDIA (\protect\hyperlink{ref-wiki:E}{2018}), a
``\textbf{esperança matemática} de uma variável aleatória é a soma do
produto de cada probabilidade de saída da experiência pelo seu
respectivo valor. Isto é, representa o valor médio `esperado' de uma
experiência se ela for repetida muitas vezes''. Matematicamente, a
Esperança de uma variável aleatória \(X\) é representada pelo símbolo
\(E[X]\), de tal forma que, pela definição dada acima, no caso de uma
variável aleatória discreta:

\[E[X] = \sum_{i = 1}^{\infty}x_ip(x_i)\]

Já para uma variável aleatória contínua, o valor esperado torna-se:

\[E[X] = \int_{-\infty}^{\infty}xf(x)dx\]

\subsection{O problema da retransformação das
variáveis}\label{o-problema-da-retransformacao-das-variaveis}

Segundo (SHEN; ZHU, \protect\hyperlink{ref-shen}{2008}, p. 552), modelos
lineares lognormais tem muitas aplicações e muitas vezes é de interesse
prever a variável resposta ou estimar a média da variável resposta na
escala original para um novo conjunto de covariantes.

Segundo Shen e Zhu(\protect\hyperlink{ref-shen}{2008}, p. 552), se
\(Z = (Z_1,\cdots, Z_n)^T\) é o vetor variável resposta de distribuição
lognormal e \(x_i = (1, x_{i1}, \cdots, x_{ip})^T\) é o vetor dos
covariantes para a observação \(i\), um modelo linear log-normal assume
a seguinte forma:

\[Y = log(Z) = X\beta + \epsilon\] onde \(X = (x_1, \cdots, x_n)^T\),
\(\beta = (\beta_0, \beta_1, \cdots, \beta_p)^T\), e
\(\epsilon = (\epsilon_1, \cdots, \epsilon_n)^T\) com
\(\epsilon_i \sim N(0, \sigma^2)\) i.i.d.(\emph{identically
independently distributed}) (SHEN; ZHU,
\protect\hyperlink{ref-shen}{2008}, pp. 552--553).

\begin{quote}
Em muitos casos, para um novo conjunto de covariantes \(x_0\), pode-se
estar interessado em prever a variável resposta em sua escala original:

\[Z_0 = e^{x_o^T\beta + \epsilon_0}\]

ou estimar a média condicional da variável resposta:

\[\mu(x_0)=E[Z_0|x_o] = e^{x_o^T\beta + \frac{1}{2}\sigma^2}\]
\end{quote}

De acordo com Shen e Zhu(\protect\hyperlink{ref-shen}{2008}, p. 553), se
\(\beta\) e \(\sigma^2\) são ambos conhecidos, então é fácil demonstrar
que o melhor estimador de \(Z_0\) é de fato \(\mu(x_0)\). Contudo, na
prática, ambos \(\beta\) e \(\sigma^2\) são desconhecidos e precisam ser
estimados para a obtenção de \(\mu(x_0)\).

Segundo Shen e Zhu (\protect\hyperlink{ref-shen}{2008}, p. 552), existem
na literatura diversos estimadores baseados em diversos métodos
inferenciais, como \textbf{ML} (\emph{Maximum Likelihood Estimator}),
\textbf{REML} (\emph{Restricted ML Estimator}), \textbf{UMVU}
(\emph{Uniformly Minimum Variance Unbiased Estimator}), além de um
estimador \textbf{REML} com viés corrigido.

Shen e Zhu(\protect\hyperlink{ref-shen}{2008}) então propõem dois novos
estimadores baseados na minimização do erro médio quadrático assintótico
e do viés assintótico.

\subsubsection{Regressão Linear}\label{regressao-linear-1}

De acordo com Duan (\protect\hyperlink{ref-Duan}{1983}, p. 606), o Valor
Esperado \(E\) de uma variável resposta \(Y\) que tenha sido
transformada em valores \(\eta\) durante a regressão linear por uma
função \(g(Y)\) \textbf{não-linear} não é igual ao valor da simples
retransformação da variável transforma pela sua função inversa
\(h(\eta) = g^{-1}(Y)\). Em outros termos(DUAN,
\protect\hyperlink{ref-Duan}{1983}, p. 606):

\[E[Y_0] = E[h(x_0\beta + \epsilon)] \ne h(x_o\beta)\]

Numa regressão log-linear, ou seja, uma regressão linear com o logaritmo
da variável dependente (\(h(\eta) = g^{-1}(\eta) = exp(\eta)\)), para
efetuar apropriadamente a retransformação das estimativas de volta a sua
escala original, precisa-se ter em conta a desigualdade mencionada na
seção \ref{esperança-matemática-ou-valor-esperado}.

Segundo (MANNING; MULLAHY, \protect\hyperlink{ref-NBERt0246}{1999}),
quando ajustamos o logaritmo natural de uma variável \(Y\) contra outra
variável \(X\) através da seguinte equação de regressão:

\[ln(Y) = \beta_0 + \beta_1X + \epsilon\]

Se o erro \(\epsilon\) é normalmente distribuído, com média zero e
desvio padrão \(\sigma^2\), ou seja, se
\(\epsilon \sim N(0, \sigma^2)\), então (DUAN,
\protect\hyperlink{ref-Duan}{1983}, p. 606; MANNING; MULLAHY,
\protect\hyperlink{ref-NBERt0246}{1999}, p. 6):

\[E[Y|X] = e^{\beta_0 + \beta_1X} \cdot E[e^\epsilon] \ne e^{\beta_0 + \beta_1X}\]

Embora o valor esperado dos resíduos \(\epsilon\) seja igual a zero, ele
está submetido a uma transformação não linear, de maneira que não
podemos afirmar que \(E[e^\epsilon] = 1\), como vimos na seção anterior.
Desta maneira, o estimador abaixo, chamado em (SHEN; ZHU,
\protect\hyperlink{ref-shen}{2008}, p. 554) de \emph{naive
back-transform estimator}, ou simplesmente \textbf{BT} não é consistente
e é enviesado, tendo viés multiplicativo de valor assintótico igual a
\(e^{-\sigma^2/2}\):

\[E[Y|X] = e^{\beta_0 + \beta_1X}\]

Segundo (SHEN; ZHU, \protect\hyperlink{ref-shen}{2008}, p. 554), ainda,
o valor de \(e^{-\sigma^2/2}\) é sempre menor do que 1(SHEN; ZHU,
\protect\hyperlink{ref-shen}{2008}, p. 554).

\begin{quote}
As a result, the BT estimator underestimates \(\mu(x_0)\), and the bias
is large when \(\sigma^2\) is large. In our study, it appears that the
BT estimator performs much worse than the other
estimators{[}\ldots{}{]}Actually, the BT estimator is more suitable for
estimating the median of Z0, which is \(exp(x_0^T\beta)\) in this case.
\end{quote}

Porém se o termo de erro \(\epsilon\) é normalmente distribuído
\(N(0,\sigma^2)\), então um estimador não-enviesado para o valor
esperado \(E[Y]\), de acordo com DUAN
(\protect\hyperlink{ref-Duan}{1983}), assume a forma vista na equação
abaixo(DUAN, \protect\hyperlink{ref-Duan}{1983}, p. 606; MANNING;
MULLAHY, \protect\hyperlink{ref-NBERt0246}{1999}, p. 2 e 6):

\[E[Y] = e^{\beta_0 + \beta_1X} \cdot e^{\frac{1}{2}\sigma^2}\]

Cabe salientar, segundo (MANNING; MULLAHY,
\protect\hyperlink{ref-NBERt0246}{1999}, p. 6), que se o termo de erro
não for i.i.d (independentes e identicamente distribuídos), mas for
homoscedástico, então:

\[E[Y|X]=s \times e^{X_0\beta}\] onde \(s = E[e^\epsilon]\).

De qualquer maneira, o valor esperado de \(Y\) é proporcional à
exponencial da previsão na escala log.

\subsubsection{Modelos
Heteroscedásticos}\label{modelos-heteroscedasticos}

Modelos heteroscedásticos não são raros, especialmente no caso de
variáveis envolvendo valores em moeda, sendo muito comum em modelos
econométricos. Em sua essência, são heteroscedásticos aqueles modelos
lineares cujo termo de erro não pode ser considerado totalmente
independente, ou seja, existe alguma função (linear ou não), tal que
\(E[e^\epsilon] = f(x)\), de modo que:

\[log(E[Y|X]) = X\beta + log(f(x))\]

É desnecessário dizer que, para estes modelos o estimador para a média é
diferente de
\(E[Y] = e^{\beta_0 + \beta_1X} \cdot e^{\frac{1}{2}\sigma^2}\), haja
vista que \(\sigma^2\) não é mais um escalar, mas uma função.

Existem diversas maneiras de se contornar este problema. Por exemplo,
através da eliminação do viés através da utilização de uma função que
modele a variância \(\sigma^2(X)\), ou através do estimador
sanduíche\footnote{ver
  \href{https://matloff.wordpress.com/2015/09/18/can-you-say-heteroscedasticity-3-times-fast/}{link}}.

Cabe ainda salientar que, para os modelos heteroscedásticos, não apenas
os erros estão comprometidos, mas também os intervalos de confiança.

\subsection{\texorpdfstring{Modelo linear generalizado
(\emph{GLM})}{Modelo linear generalizado (GLM)}}\label{modelo-linear-generalizado-glm}

De acordo com (MANNING; MULLAHY,
\protect\hyperlink{ref-NBERt0246}{1999}, pp. 3--4), um modelo linear
generalizado com uma função de ligação logarítmica estimam
\(log(E[Y|X])\) diretamente, de tal maneira que:

\[log(E[Y|X]) = X\beta\] ou \[E[Y|X] = e^{X\beta}\]

\subsection{Validação Cruzada}\label{validacao-cruzada}

Validação Cruzada ou \emph{cross-validation} é uma técnica estatística
que pode ser utilizada de diversas maneiras e consistem em dividir um
conjunto de dados em duas partições distintas, chamados de partição de
treino (\emph{training set}) e partição de teste(\emph{test set}),
utilizadas para o ajuste do modelo e para a previsão da variável
dependente, respectivamente. Os dados previstos na partição de teste são
então comparados aos valores observados.

Neste artigo efetuaremos a validação-cruzada utilizando o procedimento
chamado de \emph{delete-one procedure}, em que se retira apenas um dado
do conjunto de dados, ajusta-se um modelo e então utiliza-se este modelo
para prever o valor da variável dependente para o dado retirado (SHEN;
ZHU, \protect\hyperlink{ref-shen}{2008}, p. 564).

Para cada observação então calcula-se o seu erro quadrático
(\((Y_i - \hat{Y}_i)^2\)), utilizado para o cálculo da estatística RMSPE
(erro de previsão médio quadrático \emph{root mean squared prediction
error}), conforme expressão a seguir (SHEN; ZHU,
\protect\hyperlink{ref-shen}{2008}, p. 564):

\[(\frac{1}{n}\sum_{i = 1}^{n}(Y_i - \hat{Y}_i)^2)^{1/2}\]

\section{ESTUDO DE CASO}\label{estudo-de-caso}

Com o fim de averiguar qual estimador melhor se adequa ao procedimento
de retransformação de variáveis, aplicar-se-á um comparativo entre os
estimadores média, moda e mediana, através do uso da estatística RMSPE.

\subsection{Dados}\label{dados}

Neste estudo comparamos a precisão de diversos tipos de modelos
estatísticos (regressão linear, regressão não-linear e modelo linear
generalizado) sobre os dados disponíveis em Hochheim
(\protect\hyperlink{ref-hochheim}{2015}, pp. 21--22).

\subsection{Cálculo do RMSPE}\label{calculo-do-rmspe}

\subsubsection{Regressão linear}\label{regressao-linear-2}

Os valores ajustados com os estimadores da moda, média e mediana podem
ser vistos na tabela abaixo:

\begin{longtable}[]{@{}lrrrr@{}}
\toprule
& Y & Média & Mediana & Moda\tabularnewline
\midrule
\endhead
AP\_1 & 1.060.000 & 1.029.765 & 1.020.713 & 1.002.846\tabularnewline
AP\_2 & 510.000 & 628.132 & 622.610 & 611.712\tabularnewline
AP\_3 & 780.000 & 855.052 & 847.535 & 832.700\tabularnewline
AP\_4 & 550.000 & 736.956 & 730.478 & 717.691\tabularnewline
AP\_5 & 850.000 & 1.011.300 & 1.002.410 & 984.863\tabularnewline
AP\_6 & 300.000 & 358.594 & 355.441 & 349.220\tabularnewline
AP\_7 & 750.000 & 724.106 & 717.741 & 705.177\tabularnewline
AP\_8 & 650.000 & 657.475 & 651.695 & 640.288\tabularnewline
AP\_9 & 620.000 & 658.389 & 652.601 & 641.177\tabularnewline
AP\_10 & 740.000 & 662.002 & 656.182 & 644.696\tabularnewline
AP\_11 & 770.000 & 818.933 & 811.734 & 797.525\tabularnewline
AP\_12 & 680.000 & 702.573 & 696.397 & 684.207\tabularnewline
AP\_13 & 850.000 & 681.544 & 675.553 & 663.728\tabularnewline
AP\_14 & 420.000 & 551.781 & 546.931 & 537.357\tabularnewline
AP\_15 & 547.000 & 673.810 & 667.887 & 656.196\tabularnewline
AP\_16 & 1.600.000 & 1.413.047 & 1.400.625 & 1.376.108\tabularnewline
AP\_17 & 1.320.000 & 1.115.664 & 1.105.857 & 1.086.499\tabularnewline
AP\_18 & 615.000 & 645.338 & 639.665 & 628.468\tabularnewline
AP\_19 & 705.000 & 722.736 & 716.383 & 703.843\tabularnewline
AP\_20 & 418.000 & 435.824 & 431.993 & 424.431\tabularnewline
AP\_21 & 270.000 & 243.440 & 241.300 & 237.077\tabularnewline
AP\_22 & 418.000 & 485.426 & 481.159 & 472.736\tabularnewline
AP\_23 & 650.000 & 630.016 & 624.478 & 613.547\tabularnewline
AP\_24 & 700.000 & 774.614 & 767.805 & 754.365\tabularnewline
AP\_25 & 680.000 & 729.864 & 723.448 & 710.784\tabularnewline
AP\_26 & 420.000 & 350.336 & 347.256 & 341.178\tabularnewline
AP\_27 & 195.000 & 229.411 & 227.394 & 223.414\tabularnewline
AP\_28 & 290.000 & 279.686 & 277.228 & 272.375\tabularnewline
AP\_29 & 272.000 & 246.194 & 244.030 & 239.758\tabularnewline
AP\_30 & 430.000 & 399.634 & 396.121 & 389.187\tabularnewline
AP\_31 & 895.000 & 615.032 & 609.625 & 598.954\tabularnewline
AP\_32 & 450.000 & 454.828 & 450.830 & 442.938\tabularnewline
AP\_33 & 1.950.000 & 1.474.903 & 1.461.938 & 1.436.347\tabularnewline
AP\_34 & 2.150.000 & 2.597.848 & 2.575.011 & 2.529.937\tabularnewline
AP\_35 & 940.000 & 969.142 & 960.623 & 943.808\tabularnewline
AP\_36 & 1.400.000 & 1.334.839 & 1.323.105 & 1.299.945\tabularnewline
AP\_37 & 1.090.000 & 1.002.811 & 993.996 & 976.596\tabularnewline
AP\_38 & 1.272.000 & 999.341 & 990.556 & 973.217\tabularnewline
AP\_39 & 2.800.000 & 1.921.706 & 1.904.812 & 1.871.470\tabularnewline
AP\_40 & 1.796.000 & 2.075.621 & 2.057.374 & 2.021.361\tabularnewline
AP\_41 & 1.400.000 & 1.398.114 & 1.385.824 & 1.361.566\tabularnewline
AP\_42 & 3.000.000 & 3.306.637 & 3.277.569 & 3.220.197\tabularnewline
AP\_43 & 1.200.000 & 1.062.442 & 1.053.103 & 1.034.669\tabularnewline
AP\_44 & 800.000 & 646.536 & 640.853 & 629.635\tabularnewline
AP\_45 & 950.000 & 668.014 & 662.142 & 650.551\tabularnewline
AP\_46 & 2.061.000 & 2.267.978 & 2.248.041 & 2.208.690\tabularnewline
AP\_47 & 1.326.000 & 1.575.944 & 1.562.090 & 1.534.746\tabularnewline
AP\_48 & 850.000 & 776.375 & 769.550 & 756.079\tabularnewline
AP\_49 & 1.650.000 & 1.509.488 & 1.496.218 & 1.470.028\tabularnewline
AP\_50 & 650.000 & 834.750 & 827.412 & 812.929\tabularnewline
\bottomrule
\end{longtable}

Os valores encontrados para o erro de predição médio quadrático para
cada estimador foram: 203.939,11 para a média, 204.006,84 para a mediana
e 205.537,36 para a moda.

Como esperado, o RMSPE foi menor para a média, e maior para a moda. O
que comprova a teoria, já que o \emph{naive estimator} é enviesado com
viés conhecido de \(-\sigma^2/2\), logo a média possui viés de
\(-1,5\sigma^2\).

\subsubsection{Modelo linear
generalizado}\label{modelo-linear-generalizado}

\section{CONCLUSÃO}\label{conclusao}

Como vimos na seção \ref{sec:revisao-bibliografica}, o método clássico
de regressão linear é uma minimização do erro médio quadrático de
predição e a função de regressão \(\hat{m}_{Y;X}\) é uma equação para a
\emph{média} da população \(Y\). Considerando que são satisfeitas as
hipóteses que

\section*{REFERÊNCIAS}\label{referencias}
\addcontentsline{toc}{section}{REFERÊNCIAS}

\hypertarget{refs}{}
\hypertarget{ref-NBR1465302}{}
ABNT. \textbf{NBR 14653-2: Avaliação de bens -- parte 2: Imóveis
urbanos}. Rio de Janeiro: Associação Brasileira de Normas Técnicas,
2011.

\hypertarget{ref-Duan}{}
DUAN, N. Smearing estimate: A nonparametric retransformation method.
\textbf{Journal of the American Statistical Association}, v. 78, n. 383,
p. 605--610, 1983. Taylor \& Francis. Disponível em:
\textless{}\url{http://www.tandfonline.com/doi/abs/10.1080/01621459.1983.10478017}\textgreater{}..

\hypertarget{ref-hochheim}{}
HOCHHEIM, N. \textbf{Engenharia de avaliações - módulo básico}.
Florianópolis: IBAPE - SC, 2015.

\hypertarget{ref-NBERt0246}{}
MANNING, W. G.; MULLAHY, J. \textbf{Estimating log models: To transform
or not to transform?} Working Paper, National Bureau of Economic
Research, 1999.

\hypertarget{ref-matloff2009}{}
MATLOFF, N. S. \textbf{From algorithms to z-scores: Probabilistic and
statistical modeling in computer science}. Davis, California: Orange
Grove Books, 2009.

\hypertarget{ref-shen}{}
SHEN, H.; ZHU, Z. Efficient mean estimation in log-normal linear models.
\textbf{Journal of Statistical Planning and Inference}, v. 138, p.
552--567, 2008. Elsevier. Disponível em:
\textless{}\url{https://www.unc.edu/~haipeng/publication/emplnM1.pdf}\textgreater{}..

\hypertarget{ref-wiki:E}{}
WIKIPEDIA. Valor esperado --- Wikipedia, the free encyclopedia., 2018.
Disponível em:
\textless{}\url{https://pt.wikipedia.org/wiki/Valor_esperado}\textgreater{}..


\end{document}
